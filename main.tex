\documentclass[a4paper,10pt]{article}
\usepackage[utf8]{inputenc}
\usepackage{calc}


\newcommand{\ancienprez}{Rémi Dulong}
\newcommand{\nouveauprez}{Florian Grante}
\newcounter{annee}
\setcounter{annee}{2017}


\newcounter{anneeprec}
\setcounter{anneeprec}{\value{annee} -1}
\newcounter{anneesuiv}
\setcounter{anneesuiv}{\value{annee} +1}

%opening
\title{Dossier de passation \theannee - \theanneesuiv}
\author{\ancienprez}

\begin{document}

\maketitle

\begin{abstract}

\end{abstract}

\section{Résumé de la passation}
La passation de présidence INTech a eu lieu, comme chaque année, après l'annonce des résultats de la Coupe de France de robotique, 
évènement qui clôture notre année associative. En tant que président de l'année \theanneeprec - \theannee \space je cède donc ma place à 
\nouveauprez, élu par les membres du club.

\section{Le rôle du président}
Le président d'INTech doit s'assurer du bon fonctionnement du club. Il est responsable de l'animation des réunions internes,
des relations avec nos partenaires, nos sponsors, les autres associations du campus, l'administration des deux écoles, et de la gestion du budget.
Il doit aussi assurer la communication avec MiNET,puisque INTech est avant tout un club de l'association MiNET. \newline

Le président est chargé de donner au club l'orientation souhaitée par ses membres. Pour cela, il doit être à l'écoute des avis de ses membres.
Cela fait maintenant plus de 20 ans que nous focalisons notre activité sur la Coupe de France de robotique : si cette motivation venait à changer, il faudrait toujours 
faire le maximum pour que le club assure son rôle, à savoir la transmission de nos connaissances aux nouveaux arrivants à l'école
curieux de découvrir la robotique ou l'électronique. C'est ce qui doit être la motivation première du président du club.

\end{document}
