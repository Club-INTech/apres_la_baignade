\documentclass[a4paper,10pt]{article}
\usepackage[utf8]{inputenc}
\usepackage{calc}
\usepackage{graphicx}
\usepackage{textcomp}
\usepackage[francais]{babel}

\newcommand{\ancienprez}{Florian Grante}
\newcommand{\nouveauprez}{Victor Védie}
\newcounter{annee}
\setcounter{annee}{\year}


\newcounter{anneeprec}
\setcounter{anneeprec}{\value{annee} -1}
\newcounter{anneesuiv}
\setcounter{anneesuiv}{\value{annee} +1}

%opening
\title{Dossier de passation \theannee~-~\theanneesuiv}
\author{\ancienprez}

\begin{document}

\maketitle

\begin{center}
\includegraphics[scale=0.3]{logo-grand.png}
\end{center}


\section{Résumé de la passation}

La passation de présidence INTech a eu lieu, comme chaque année, après l'annonce des résultats de la Coupe de France de robotique, 
événement qui clôture notre année associative. En tant que président de l'année \theanneeprec~-~\theannee \space je cède donc ma place à 
\nouveauprez, élu par les membres du club. 

\section{Le rôle du président}

Le président d'INTech doit s'assurer du bon fonctionnement du club. Il est responsable de l'animation des réunions internes,
des relations avec nos partenaires, nos sponsors, les autres associations du campus, l'administration des deux écoles, et de la gestion du budget.
Il doit aussi assurer la communication avec MiNET,puisque MiNET est avant tout un club de l'association INTech. \newline

Le président est chargé de donner au club l'orientation souhaitée par ses membres. Pour cela, il doit être à l'écoute de l'avis de chacun d'entre eux.
Cela fait maintenant plus de 25 ans que nous focalisons notre activité sur la Coupe de France de robotique : si cette motivation venait à changer, il faudrait toujours 
faire le maximum pour que le club assure son rôle, à savoir la transmission de nos connaissances aux nouveaux arrivants à l'école,
curieux de découvrir la robotique ou l'électronique. Cela doit demeurer la motivation première du président du club en toutes circonstances.\newline

Il devra également veiller à l'entretien des différents partenariats. Nous possédons de très précieux partenaires, avec lesquels nous tentons au maximum 
de garder une relation gagnant-gagnant. Cela est valable pour les partenaires extérieurs à l'école, mais aussi pour les partenariats et arrangements internes 
à l'école, et en particulier envers GATE\texttrademark.\newline

En fonction des centres d'intérêt de son équipe de 2A, mais aussi des nouveaux arrivants 1A, il pourra décider de continuer ou d'interrompre les projets commencés 
lors des années précédentes, ou d'en lancer de nouveaux.\newline

Il faut également rappeler que notre local regorge de matériel potentiellement dangereux s'il n'est pas utilisé avec les précautions et règles de sécurité nécessaires.
Le président du club doit s'assurer que ces règles soient respectées, en les rappelant aux membres aussi souvent que nécessaire, mais aussi en montrant l'exemple.
Ce dernier point s'applique également à tous les 2A membres du club, et ce afin de transmettre les bonnes pratiques aux 1A.

\section{Conclusion}

Le président d'INTech sera donc \nouveauprez \space pour l'année \theannee~-~\theanneesuiv. Je lui souhaite un bon mandat, et espère qu'il saura tirer tout l'épanouissement 
inhérent à ce poste.

\end{document}
